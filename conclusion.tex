In present work, the theoretical approach to the photoelectron spectra of molecules of arbitrary symmetry is attempted uniting the three essential concepts which can be considered being state-of-the-art by themselves but have not been tried together. In the heart of the method is the electronic structure approach utilizing optimally-tuned range-separated density functional which is obtained in a fully automated self-consistent procedure. With this, the asymptotic behaviour of electron density is corrected, due to the elimination of the self-interaction error, which leads to notable improvement of various molecular properties. By construction, it is particularly suited for predicting valence photoelectron spectra due to the improvement of the quasiparticle binding energies.
Another ingredient is the frequency-domain Dyson orbital formalism. Due to the neglect of correlation between bound- and photo-electrons it allows for the efficient computational scheme, which can be applied to much larger molecular systems than the approaches, where bound and unbound electrons are treated on the same footing. This makes this technique very attractive for applications in chemistry, bio- and solid-state physics where large-scale objects usually occur. Nevertheless, despite its efficiency, it includes many-body effects provided by the underlying quantum-chemical method, thus, taking into account combination photoelectron transitions and further correlation and relaxation effects.
The most ambitious objective has been to implement a rigorous finite element representation of the photoelectron wave function experiencing, in general, intricate molecular electrostatic potential. To ensure the correct asymptotic behavior the finite element scheme was employed. It is well established for exterior acoustics but has not yet been extensively applied to quantum-mechanical problems.
The electronic structure protocol has been tested for four different molecular systems. It has demonstrated good agreement with experiment for the benzene being a prototypical organic conjugated molecule The same almost quantitative agreement have been seen for water. However, for linear CO$_2$ molecule a breakdown of the picture has been observed, which apparently can be connected to conceptual deficiency of Kohn-Sham DFT being a single-configurational method. Moreover, the triplet stability of the DFT solution needs to be considered while choosing optimal parameters, to get reliable results. In general, one can expect that the first ionisation transitions corresponding to the lowest binding energies are reproduced fairly well due to their consistent nature ensured by optimal tuning procedure. For higher binding energies, the agreement could be worse demonstrating overestimation of the transition energy.
The major effort has been spent on implementing and testing the finite/infinite element method to describe the free electron functions for an arbitrary complex molecule. Although being a flexible approach it suffers from a number of numerical and conceptual problems. Importantly, the requirements to accurately describe energies of continuum functions are opposite to those needed for reliable predictions of cross sections. Thus, improving the flexibility of the computational setup leads to the free electron functions with high angular momentum, which do not contribute to the intensity. Since kinetic energies of photoelectrons varying in wide ranges require substantially different conditions, this prevents one from calculating systematic dependencies of cross sections. Nevertheless, the idea of such an (in)finite element method is very appealing and further investigations of the parameter space might be useful.
As a final conclusion, the usage of the analytic Coulomb waves with the developed protocol is encouraged. In general, they correspond to the systematic improvement of the calculated intensities over the Koopmans' approach and sudden approximation.

